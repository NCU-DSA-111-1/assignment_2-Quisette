\section{先備知識}
    由於本專案大量的使用日本將棋之日文術語用於命名變數,為求程式之可讀性,在此列出可能會使用到之術語:
	\subsection{棋駒}
	\begin{table}[h]
	    \centering
	    \begin{tabular}{c|c|c|c}
	        棋駒名稱 & 日文簡寫 & 羅馬拼音 & 羅馬簡寫\\
	        \hline
	        步兵 & 步 & fuhyou & fu\\
	        香車 & 香 & kyousha & kyo\\
	        桂馬 & 桂 & keima & kei\\
	        銀將 & 銀 & ginshou & gin\\
	        金將 & 金 & kinshou & kin\\
	        飛車 & 飛 & hishya & hi\\
	        角行 & 角 & kakugyo & kaku\\
	        玉將 & 玉 & gyokushou & gyoku\\
	         \hline
	        と金 & と & tokin & to\\
	        成香 & 杏 & narikyo & narikyo\\
	        成桂 & 圭 & narikei & narikei\\
	        成銀 & 全 & narigin & narigin\\
	        竜王 & 竜 & ryuuou & ryu\\
	        竜馬 & 馬 & ryuuma & uma\\
	    \end{tabular}
	    \caption{棋駒名稱及常用羅馬拼音對照表}
	    \label{tab:my_label}
	\end{table}
	\subsection{術語解釋}
	於此章節,主要介紹於此程式可能會使用到的術語。該術語將會長時間
	\begin{table}[]
	    \centering
	    	\begin{tabular}{c|c|c|l}
	        術語 & 日文 & 羅馬拼音 & 定義\\
	        \hline
	        成變 & なる  & naru  & 友方棋駒到達敵陣三段目時之棋駒升級\\
	        持駒& もちこま & mochikoma& 捕獲的對方持駒列表\\
	        盤面& ばんめん & bannmenn&將棋遊戲盤面\\
	        棋譜 & きふ & kifu& 紀錄對局紀錄之檔案\\
	        手番 & てばん & teban & 現在盤面輪到的對局者\\
	        先手 & せんて & sente&     開局先走棋的對局者(奇數手)\\
	        後手 & ごて & gote& 開局後走棋的對局者(偶數手)\\
	       
	    \end{tabular}
	    \caption{術語名稱及常用羅馬拼音對照表}
	    \label{tab:my_label}
	\end{table}

	

	\subsection{棋譜 / 遊戲紀錄格式}
	In this section, we will introduce how the kifus (or game records ) are stored in Japanese Shogi Association as well as Computer Shogi Association.
	
	\subsubsection{棋譜格式}


    \lstset{escapeinside=``}
    \subsubsection{KIF 記譜格式}
    KIF記譜格式為傳統記譜外最普及的將棋棋譜格式,因篇幅所現不仔細講解,並列格式如下:
    \lstinputlisting[language=C]{kifu/kif.txt}
	\subsubsection{CSA 記譜格式}
	根據計算機將棋協會(コンピュータ将棋協会, Computer Shogi Association)所制定之計算機通用棋譜紀錄格式\cite{CSA},在此列出CSA棋譜之記錄格式:
	
	
	
	\paragraph{盤面} 紀錄當時棋局的盤面。其要點如下所示:
	\begin{itemize}
	
	    \item 各棋駒由下表賦予其英文代號:
	    \begin{table}[h]
	        \centering
	        \begin{tabular}{c|c|c|c|c|c|c|c}
	           步  &香 & 桂 &銀&金&飛&角&玉/王 \\
	           \hline
	             FU  &KY & KE &GI&KI&HI&KA&OU\\
	            \hline
	            と  &杏 & 圭 &全&&竜&馬\\
	            \hline
	            TO &NY & NK &NG& &RY&UM\\
	        \end{tabular}
	        \caption{KIF代號對照表}
	        \label{tab:my_label}
	    \end{table}
	    \item 每個CSA檔案可包含一(可選的)盤面作為初始盤面。若無指定,將會預設為一般的初始盤面。

	    \item 若要記錄持駒,可使用如 \verb|P[先後手]00[棋駒代號]| 之方式。 
	    
	\end{itemize}
	\paragraph{棋譜} 移動的棋譜將以以下物件生成:
	    \begin{itemize}
	        \item 手番: 以\verb|+|或是\verb|-| 表示。
	        \item 初始座標:將棋駒移動前的座標以[筋][段]之方式表示。
	        \item 結束座標:將棋駒結束移動座標以[筋][段]之方式表示。
	    \end{itemize}
	\paragraph{終局} 終局時,可以 \verb|%TORYO| 表記棋局結束。
	以下為標準的CSA 棋譜表示法:
	\lstinputlisting[]{kifu/csa.txt}
	\subsubsection{SFEN 記譜格式\cite{SFEN}}
	\label{chap:sfen}
	SFEN 格式全稱為 \textit{Shogi Forsyth-Edwards Notation} ,係仿製西洋棋之 \textit{Forsyth-Edwards Notation} \cite{FEN} 所定義的盤面標示格式。該類型之棋譜可用於紀錄當前盤面,且無須紀錄該盤面自初手開始的棋局變化。所需之欄位如下所示:
	\paragraph{盤面}
	紀錄當時棋局的盤面。其要點如下所示:
	\begin{itemize}
	    \item 棋譜輸入自左到右,由上到下;對應於傳統將棋座標中,自9一開始到1九結束。
	    \item 各棋駒由下表賦予其英文代號:
	    \begin{table}[h]
	        \centering
	        \begin{tabular}{c|c|c|c|c|c|c|c}
	           步  &香 & 桂 &銀&金&飛&角&玉/王 \\
	           \hline
	             P  &L & N &S&G&R&B&K
	        \end{tabular}
	        \caption{SFEN代號對照表}
	        \label{tab:my_label}
	    \end{table}
	    \item 若該棋駒為昇變棋駒,則於該代號前加上一個 \verb|+| 號標記。
	    \item 無棋駒的連續 $n$ 個空格以數字$n$表示。
	    \item 一行以九個棋駒為限,行與行間用 \verb|/| 區隔。若令一行中有 $m$ 個英文字元與 $n$ 個數字,分別為$n_1, n_2 \cdots$,則每一行有以下限制:
	    $$
	    m + \sum n_i = 9
	    $$
	\end{itemize}
	\paragraph{手番}記錄該盤面下一手的所屬。
	\paragraph{持駒}記錄先後手的持駒。大寫的代號為先手持駒,小寫則為後手持駒。所有的持駒將會以無空白字串儲存。若一方持有 $n$ 個相同持駒,則在對應的代號上加上數字 $n$ 。
	\paragraph{手數}紀錄當時的手數(自棋局開始後經過的棋步數量)。
	
    \subsection{初手局面}
    此為將棋棋局的最初局面,如下圖:
    \begin{figure}[h]
        \centering
          \shogiban{\hirate}
        \caption{棋局初手局面}
        \label{fig:my_label}
    \end{figure}
    該盤面可使用SFEN記譜表示為
    \begin{lstlisting}
lnsgkgsnl/1r5b1/ppppppppp/9/9/9/PPPPPPPPP/1B5R1/LNSGKGSNL b - 1
\end{lstlisting}