	\section{程式}
	於此章節,我們在此簡要敘述本程式使用到的函式及預定義的常數。
	\subsection{實作功能}
	\begin{itemize}
	    \item 棋駒移動
	    \item 棋駒昇變
	    \item 駒台與持駒打入
	    \item 棋譜輸出
	    \item 輸出一般網站可使用之棋譜(sfen)
	\end{itemize}
	\subsection{預定義常數}
	\lstinputlisting[language=C]{src/const.c}
	\subsection{結構}
	於此章節,我們定義此程式使用的結構體如下:
\lstinputlisting[language=C]{src/structures.c}	
	\subsection{函式}
\lstinputlisting[language=C]{src/functions.c}
	

	\subsection{編譯方式}
	    請下載並開啟本專案資料夾,並輸入以下指令:
	    \begin{lstlisting}[language=sh,caption=Compile Parameters]
$ make  
$ ./main	     \end{lstlisting}

\subsection{棋譜輸出格式}
於本專案中,遊戲過程將於退出時紀錄於\verb|./sfenlist|。該格式之讀法請參照\autoref{chap:sfen},以一行為單位為一手的盤面。範例輸出如下:
\begin{lstlisting}
lnsgkgsnl/1r5b1/ppppppppp/9/9/9/PPPPPPPPP/1B5R1/LNSGKGSNL b - 1
lnsgkgsnl/1r5b1/ppppppppp/9/9/2P6/PP1PPPPPP/1B5R1/LNSGKGSNL w - 2
lnsgkgsnl/1r5b1/pppppp1pp/6p2/9/2P6/PP1PPPPPP/1B5R1/LNSGKGSNL b - 3
lnsgkgsnl/1r5+B1/pppppp1pp/6p2/9/2P6/PP1PPPPPP/7R1/LNSGKGSNL w B 4
lnsgkg1nl/1r5s1/pppppp1pp/6p2/9/2P6/PP1PPPPPP/7R1/LNSGKGSNL b Bb 5
lnsgkg1nl/1r5s1/pppppp1pp/6p2/4B4/2P6/PP1PPPPPP/7R1/LNSGKGSNL w b 6
\end{lstlisting}

\subsection{程式使用方式}



\subsubsection{開始新局面}
請執行以下指令
\begin{lstlisting}[language=sh,caption=Compile Parameters]
$ ./main	     
\end{lstlisting}
以開啟新局面。
進入後,系統將會提示輸入先手方之棋步,請依照以下規則輸入:[初始座標][結束座標][棋駒羅馬簡寫],座標之(1,1)位於棋盤先手方之左上角。\\
範例如下:
\begin{lstlisting}[,caption=Examples of Inputs]
77 76 fu
33 34 fu
88 22 kaku
31 22 gin 
00 55 kaku //`持駒打入,初始座標為00`
\end{lstlisting}
若棋駒可昇變,將會提示
\begin{lstlisting}[]
`<<成りますか?>>(naru, narazu)`
\end{lstlisting}
請輸入\verb|naru|以昇變或\verb|narazu|以保持為原棋駒。\\
若要悔棋,可以輸入
\begin{lstlisting}[]
revert
\end{lstlisting}
;若要退出,請輸入
\begin{lstlisting}[]
quit
\end{lstlisting}
。退出之後,此程式會紀錄該次棋局之棋譜,存放於
\begin{lstlisting}[]
./kifu.sfenlist
\end{lstlisting}
存放,待回放使用。
\subsubsection{單行sfen棋譜讀入}
請執行以下指令
\begin{lstlisting}[language=sh,caption=Compile Parameters]
$ ./main --readsfen ${SFENFILE_LOCATION}	     
\end{lstlisting}
以使用sfen格式繼續中斷的棋局,並繼續遊玩。
您可以使用 \\https://lishogi.org/analysis/ 以生成一個新盤面。
\subsubsection{棋譜瀏覽}
本棋譜瀏覽功能僅限讀入本程式之 \verb|sfenlist| 檔案。請使用
\begin{lstlisting}[language=sh]
$ ./main --replay ${SFENLIST_FILE_LOCATION}	     
\end{lstlisting}
並使用 \verb|f, b| 前後瀏覽。退出時按下 \verb|q| 即可。 